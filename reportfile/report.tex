\documentclass[a4paper,12pt,oneside]{report}

% ------------------------------------------------
% LANGUAGE & FONTS
% ------------------------------------------------
\usepackage[british]{babel}
\usepackage[T1]{fontenc}
\usepackage[utf8]{inputenc}
\usepackage{newtxtext,newtxmath} % Times New Roman-like fonts for pdflatex

% ------------------------------------------------
% PAGE GEOMETRY
% ------------------------------------------------
\usepackage[a4paper,
    left=3.5cm,
    right=1.25cm,
    top=2.5cm,
    bottom=1.25cm
]{geometry}

% ------------------------------------------------
% LINE SPACING & INDENTATION
% ------------------------------------------------
\usepackage{setspace}
\doublespacing
\setlength{\parindent}{1.25cm}

% ------------------------------------------------
% CHAPTER, SECTION, SUBSECTION FORMATTING
% ------------------------------------------------
\usepackage{titlesec}
% Chapter Title - 16pt, Bold, Capital, Centered
\titleformat{\chapter}[block]
  {\centering\normalfont\bfseries\fontsize{16}{20}\selectfont}
  {\thechapter.}{1em}{\MakeUppercase}

% Section - 14pt, Bold
\titleformat{\section}
  {\normalfont\bfseries\fontsize{14}{16}\selectfont}{\thesection}{1em}{}

% Subsection - 12pt, Bold
\titleformat{\subsection}
  {\normalfont\bfseries\fontsize{12}{14}\selectfont}{\thesubsection}{1em}{}

% ------------------------------------------------
% CAPTION SETTINGS
% ------------------------------------------------
\usepackage{caption}
\captionsetup[figure]{
    font={it,small},
    labelfont={bf},
    labelsep=period,
    justification=centering,
    name=Figure
}
\captionsetup[table]{
    font={it,small},
    labelfont={bf},
    labelsep=period,
    justification=centering,
    name=Table,
    position=top
}

% ------------------------------------------------
% NUMBERING FORMAT
% ------------------------------------------------
\numberwithin{equation}{chapter}
\numberwithin{figure}{chapter}
\numberwithin{table}{chapter}

% ------------------------------------------------
% OTHER PACKAGES
% ------------------------------------------------
\usepackage{graphicx}
\usepackage{amsmath}
\usepackage{listings}
\usepackage{float}
\usepackage{makeidx}
\usepackage{hyperref}
\usepackage{xcolor}
\usepackage{microtype}

\lstset{
    basicstyle=\ttfamily\footnotesize,
    breaklines=true
}

% ------------------------------------------------
% TITLE PAGE
% ------------------------------------------------
\begin{document}

\begin{titlepage}
\begin{center}
    \Large\textbf{FOUR WEEK TRAINING REPORT}\\[3mm]
    \large{at}\\[3mm]
    \Large\textbf{Academic Advancement of Information Technology, Mohali}\\[6mm]
    \normalsize{SUBMITTED IN PARTIAL FULFILLMENT OF THE REQUIREMENTS FOR THE AWARD OF DEGREE OF}\\[3mm]
    \Large\textbf{BACHELOR OF TECHNOLOGY}\\[3mm]
    \large{in Computer Science and Engineering}\\[9mm]
    \includegraphics[height=6cm]{gndeclogo.png}\\[6mm]
    \large{JUNE–JULY 2025}\\[9mm]
    \textbf{SUBMITTED BY:}\\
    NAME: Ayush Mehta\\
    UNIVERSITY ROLL NO.: 2302489
\end{center}

\vspace{10mm}
\begin{center}
    \textbf{Department of Computer Science and Engineering}\\
    \textbf{Guru Nanak Dev Engineering College}\\
    \textbf{Ludhiana, 141006}
\end{center}
\end{titlepage}

% ------------------------------------------------
% CERTIFICATE PAGE (TO BE INSERTED FROM COMPANY)
% ------------------------------------------------
\newpage
\begin{center}
    \large\textbf{CERTIFICATE}\\[6mm]
    \includegraphics[width=\textwidth, keepaspectratio]{certificate.jpg}
\end{center}

% ------------------------------------------------
% CANDIDATE’S DECLARATION
% ------------------------------------------------
\newpage
\begin{center}
    \large\textbf{CANDIDATE'S DECLARATION}
\end{center}

I, \textbf{Ayush Mehta}, hereby declare that I have undertaken four-week Web Development training from \textbf{Academic Advancement of Information Technology, Mohali} during the period from 26 June 2025 to 26 July 2025 in partial fulfillment of the requirements for the award of the degree of \textbf{B.Tech. (Computer Science and Engineering)} at \textbf{Guru Nanak Dev Engineering College, Ludhiana}. The work presented in this training report is an authentic record of my training.

\vspace{15mm}
\noindent
\textbf{(Ayush Mehta)}\\
Roll No.: 2302489\\[10mm]
The four week industrial training Viva--Voce Examination of \rule{4cm}{0.4pt} has been held on \rule{3cm}{0.4pt} and accepted.

\vspace{45mm}

\noindent
\begin{tabular}{p{0.45\textwidth}p{0.45\textwidth}}
\centering\textbf{Signature of External Examiner} & \centering\textbf{Signature of Internal Examiner} \\
\end{tabular}

% ------------------------------------------------
% ABSTRACT
% ------------------------------------------------
\newpage
\begin{center}
    \large\textbf{ABSTRACT}
\end{center}

This report summarizes the four-week industrial training in Web Development undertaken at \textbf{Academic Advancement of Information Technology (A2IT), Mohali}. The training primarily focused on learning the fundamentals of front-end web technologies, including \textbf{HTML} and \textbf{CSS}, along with an introductory understanding of \textbf{JavaScript}. 

As a beginner to web development, this training provided me with a strong foundation in creating structured, styled, and responsive web pages. The sessions covered essential concepts of website design and layout, enabling me to understand how the various components of a web application interact. 

Towards the end of the training, I developed a small project—a \textbf{Scientific Web Calculator}—which allowed me to apply the knowledge gained during the sessions. Although simple, this project served as a practical exercise to consolidate the learning outcomes. Overall, the training proved to be an invaluable starting point for my journey into web development and helped me build confidence in working with core web technologies.

% ------------------------------------------------
% ACKNOWLEDGEMENT, CONTENTS, ETC.
% ------------------------------------------------

\newpage
\begin{center}
    \large\textbf{ACKNOWLEDGEMENT}
\end{center}

I express my deepest sense of gratitude to \textbf{Dr.~Sehijpal Singh}, Principal, Guru Nanak Dev Engineering College, Ludhiana, for providing the necessary facilities and environment for carrying out the training successfully. I am equally thankful to \textbf{Dr.~Kiran Jyoti}, Head, Department of Computer Science and Engineering, for her valuable support, motivation, and guidance during the course of the training.

I am sincerely thankful to \textbf{Mr.~Jaswant Singh} and \textbf{Ms.~Kuljit Kaur}, Training Coordinators, Department of Computer Science and Engineering, for their constant guidance, encouragement, and for providing valuable instructions regarding the preparation of this report and the training documentation.

I would also like to express my heartfelt thanks to the management and staff of \textbf{Academic Advancement of Information Technology (A2IT), Mohali} for providing me the opportunity to undergo industrial training. I extend my sincere appreciation to \textbf{Ms.~Payal Karn} for her continuous guidance, insightful lectures, and support throughout the training period. I am also thankful to \textbf{Mr.~Rajeev}, Director of Technology, A2IT, for facilitating the overall training program, and to \textbf{Ms.~Harpreet Kaur}, Senior Manager (Human Resources), A2IT, for her assistance in administrative and certification matters.

Lastly, I extend my gratitude to all my faculty members, friends, and family who directly or indirectly helped me during this training and in preparing this report.

\vspace{10mm}
\noindent
\textbf{(Ayush Mehta)}\\
Roll No.: 2302489

\tableofcontents
\listoffigures
\listoftables

% ------------------------------------------------
% CHAPTERS
% ------------------------------------------------
\newpage
\chapter{INTRODUCTION}

\section{BACKGROUND}

The field of web development has rapidly evolved over the past few decades, becoming one of the most dynamic and essential domains in computer science and information technology. The internet, once a medium for static information display, has transformed into a highly interactive platform enabling businesses, educational institutions, and individuals to connect globally.

In the early stages of web design, developers primarily relied on simple \textbf{HTML (HyperText Markup Language)} to create basic webpages. As the need for visual appeal and structure grew, \textbf{CSS (Cascading Style Sheets)} was introduced, allowing developers to separate content from design and apply consistent styling across webpages. Over time, the demand for interactivity gave rise to \textbf{JavaScript}, which added dynamic features such as form validation, animations, and responsive user interfaces.

Today, web development encompasses a wide variety of technologies and frameworks designed to enhance both performance and user experience. However, understanding the fundamental building blocks—HTML, CSS, and JavaScript—remains crucial for every aspiring web developer. Mastery of these core technologies lays the foundation for advanced front-end and full-stack development. 

This training program was designed with beginners in mind, focusing on practical, hands-on exposure to these core technologies. As a newcomer to web development, this four-week program provided me the opportunity to move from zero prior experience to confidently developing simple, well-structured, and visually appealing webpages.

\section{OBJECTIVE OF THE TRAINING}

The primary objective of the four-week industrial training in Web Development was to equip participants with essential skills and foundational understanding of front-end technologies. The training aimed to foster practical learning through continuous implementation rather than purely theoretical study.

The key objectives of the training were as follows:

\begin{itemize}
    \item To understand and implement semantic HTML for structured and accessible web pages.
    \item To learn CSS fundamentals for styling, layout design, and responsive web development.
    \item To gain familiarity with JavaScript basics including variables, data types, operators, functions, conditionals, loops, and events.
    \item To understand the Document Object Model (DOM) and how JavaScript interacts with HTML elements.
    \item To design responsive and accessible web interfaces using modern CSS techniques and frameworks.
    \item To develop small-scale, functional web projects demonstrating integration of HTML, CSS, and JavaScript.
    \item To build a simple \textbf{Scientific Web Calculator} as a final project, consolidating the concepts learned throughout the training.
\end{itemize}

The training was conducted in a structured, progressive manner—beginning with the basics of markup and styling and culminating in interactive page design and scripting. This approach helped bridge the gap between conceptual understanding and practical application.

\section{OVERVIEW OF WEB DEVELOPMENT TRAINING}

The web development training covered the three core technologies that form the backbone of front-end development:

\begin{itemize}
    \item \textbf{HTML (HyperText Markup Language):} Used to structure web content with elements like headings, paragraphs, tables, forms, images, and multimedia.
    \item \textbf{CSS (Cascading Style Sheets):} Used to style and visually enhance HTML elements, enabling layout control, color schemes, typography, and responsive design.
    \item \textbf{JavaScript:} A lightweight scripting language used to add interactivity, handle user inputs, and manipulate the Document Object Model dynamically.
\end{itemize}

Throughout the training, emphasis was placed on clean, semantic coding practices and the use of external style sheets for maintainability. Responsive design techniques, accessibility considerations, and code validation were also highlighted. 

In the final phase of training, the concepts were integrated through the creation of a \textbf{Scientific Web Calculator} project. This project demonstrated the application of HTML structure, CSS styling, and JavaScript functionality to create a practical and interactive web-based tool.

\section{IMPORTANCE OF WEB DEVELOPMENT IN THE MODERN ERA}

Web development is one of the most sought-after skills in the modern digital age. Almost every organization—from startups to global enterprises—relies on web applications to deliver products, services, and information to users worldwide. Understanding the fundamentals of web technologies is essential not only for computer science students but also for anyone interested in digital innovation.

Some of the key reasons for the importance of web development include:

\begin{itemize}
    \item \textbf{Universal Accessibility:} Websites serve as globally accessible platforms for information, communication, and commerce.
    \item \textbf{Career Relevance:} Proficiency in HTML, CSS, and JavaScript is a foundational skill set required for many modern software roles.
    \item \textbf{Creative and Analytical Balance:} Web development uniquely combines logical problem-solving with visual and creative design.
    \item \textbf{Scalability and Flexibility:} Websites and web apps can be scaled easily across devices and platforms, making them cost-effective and efficient.
    \item \textbf{Continuous Growth:} The web ecosystem is constantly evolving, with new tools, frameworks, and best practices emerging regularly.
\end{itemize}

Through this training, I have gained a beginner-level yet substantial understanding of how web pages are structured, styled, and made interactive. This foundation paves the way for future exploration into advanced frameworks and backend technologies.

\section{SCOPE OF TRAINING}

The training covered both theoretical and practical aspects of front-end development, emphasizing implementation-based learning. The scope of work and skill development can be summarized as follows:

\textbf{HTML and CSS Development:}
\begin{itemize}
    \item Creation of semantic HTML structures including tables, lists, and forms.
    \item Integration of multimedia elements such as images, audio, and video.
    \item Application of CSS for page layout, typography, spacing, and visual aesthetics.
    \item Implementation of responsive design principles using relative units and media queries.
    \item Understanding the box model, selectors, and cascading hierarchy.
\end{itemize}

\textbf{JavaScript Fundamentals:}
\begin{itemize}
    \item Learning basic syntax, data types, and operators.
    \item Using control flow structures like conditionals and loops.
    \item Understanding functions, events, and DOM manipulation.
    \item Developing small scripts for user interaction and dynamic page behavior.
\end{itemize}

\textbf{Final Project:}
\begin{itemize}
    \item Designing and developing a simple \textbf{Scientific Web Calculator}.
    \item Implementing user interaction through event handling.
    \item Managing form inputs, mathematical operations, and display updates via JavaScript.
    \item Styling the interface using CSS to ensure readability and usability.
\end{itemize}

Overall, the training provided a strong introduction to web development principles and practices, enabling me to create structured, styled, and functional web applications independently.


\newpage
\chapter{TRAINING WORK UNDERTAKEN}

The four-week training in Web Development at \textbf{Academic Advancement of Information Technology (A2IT), Mohali} was aimed at introducing the fundamental concepts and practices of modern web design and development. The training was primarily focused on the core building blocks of front-end technologies—\textbf{HTML}, \textbf{CSS}, and the introductory concepts of \textbf{JavaScript}. Over the course of the training, emphasis was placed on understanding the logical structure of web pages, styling principles, and the integration of interactivity to create a complete and responsive web experience.

\vspace{5mm}
\noindent
\section{WEEK 1 – INTRODUCTION TO HTML AND WEB STRUCTURE}

The first week of the training focused on building a foundational understanding of web technologies through \textbf{HTML (HyperText Markup Language)} and basic \textbf{CSS (Cascading Style Sheets)} concepts. The objective was to enable learners to design and structure fully functional static web pages while adhering to semantic and syntactic correctness.

\textbf{Topics Covered:}
\begin{itemize}
    \item Introduction to HTML and its importance in defining the structure of web documents.
    \item Document hierarchy using \texttt{<!DOCTYPE html>}, \texttt{<html>}, \texttt{<head>}, and \texttt{<body>} tags.
    \item Practice on nested ordered and unordered lists with different bullet styles such as Roman numerals, alphabets, circle, square, and disc types.
    \item Development of structured and multi-level list hierarchies emphasizing indentation and readability.
    \item Creation of HTML tables using \texttt{<table>}, \texttt{<tr>}, \texttt{<th>}, and \texttt{<td>} along with attributes like \texttt{rowspan} and \texttt{colspan}.
    \item Embedding of multimedia content such as videos and maps using the \texttt{<embed>} and \texttt{<audio>} tags.
    \item Integration of hyperlinks, images, and tables to create a prototype introductory website.
    \item Introduction to basic CSS styling techniques including internal, external, and inline CSS.
    \item Use of box model properties—margins, padding, borders, and box-sizing—to control page layout.
    \item Application of selectors, IDs, and classes to style specific HTML elements effectively.
    \item Introduction to \texttt{<div>} containers, icon usage, and the difference between absolute and relative sizing units (\textit{px}, \textit{\%}, and \textit{vh}).
\end{itemize}

By the end of Week 1, participants could construct and style well-structured multi-page websites using semantic HTML and basic CSS. They demonstrated proficiency in organizing content hierarchically, applying styles, and validating markup, thereby laying a strong foundation for upcoming responsive and interactive web design tasks.

\vspace{5mm}
\noindent
\section{WEEK 2 – INTRODUCTION TO CSS AND PAGE STYLING}

The second week expanded on styling principles through advanced \textbf{CSS} techniques and responsive layout design. The primary focus was on creating aesthetically appealing, device-responsive web pages that maintained consistency and clarity across various screen sizes.

\textbf{Topics Covered:}
\begin{itemize}
    \item Implementation of layouts using background images, overlays, and positioned elements.
    \item Application of color theory, typography, and branding elements in layout design.
    \item Development of multi-section pages such as product landing pages, promotional banners, and service websites.
    \item Introduction to modern layout tools — \textbf{Flexbox} and \textbf{CSS Grid} — for flexible, adaptive page structures.
    \item Understanding Flexbox properties: \texttt{justify-content}, \texttt{align-items}, \texttt{flex-wrap}, and \texttt{align-self}.
    \item Implementation of responsive design principles using \texttt{@media} queries for different device widths.
    \item Creation of navigation menus and responsive headers using Flexbox alignment and spacing.
    \item Practice with semantic HTML structure for clean and accessible markup.
    \item Styling and animation of buttons, hover effects, and interactive user interface components.
    \item Integration of Font Awesome icons (\texttt{fa-phone}, \texttt{fa-paper-plane}, \texttt{fa-cart-shopping}, \texttt{fa-bars}) for improved UI clarity.
    \item Construction of the “\textbf{Vegefoods}” clone project involving responsive navigation, sticky header, and animated hero section using \texttt{@keyframes}.
    \item Utilization of Google Fonts and CSS transitions to enhance text presentation and interactivity.
\end{itemize}

Through repeated hands-on exercises, learners gained confidence in designing flexible, well-structured interfaces adaptable to both desktop and mobile displays. By the end of this phase, they were proficient in combining creative design elements with responsive layout logic to produce professional-grade webpages.

\vspace{5mm}
\noindent
\section{WEEK 3 – INTRODUCTION TO JAVASCRIPT AND INTERACTIVITY}

The third week transitioned from static design to dynamic web functionality through the introduction of \textbf{JavaScript}. Participants explored scripting fundamentals and practiced creating interactive features that respond to user actions in real time.

\textbf{Topics Covered:}
\begin{itemize}
    \item Introduction to JavaScript syntax, data types, and variables using \texttt{var}, \texttt{let}, and \texttt{const}.
    \item Execution of JavaScript using Node.js for console output and browser-based interaction via the DOM.
    \item Understanding of arithmetic, comparison, and logical operators, and their use in expressions.
    \item String operations such as concatenation, slicing, splitting, and case manipulation.
    \item Use of conditional statements (\texttt{if-else}, \texttt{switch}) and iterative loops (\texttt{for}, \texttt{forEach}) for program flow control.
    \item Implementation of arrays and array methods such as \texttt{sort()}, \texttt{slice()}, \texttt{splice()}, and \texttt{reverse()}.
    \item DOM manipulation techniques using \texttt{document.querySelector()}, \texttt{innerText}, and \texttt{addEventListener()}.
    \item Building interactive interfaces that dynamically update content in response to user events.
    \item Introduction to object-oriented programming concepts including object literals, ES6 classes, constructors, and methods.
    \item Understanding the use of \texttt{this} keyword within class methods and creation of multiple object instances.
    \item Practice tasks such as building card-based layouts, hover effects, and dynamic content animations purely with JavaScript and CSS.
\end{itemize}

By the completion of Week 3, participants were capable of combining HTML, CSS, and JavaScript to produce interactive and visually engaging web pages. They learned to create reusable components, handle user inputs, and manage DOM elements efficiently, marking the beginning of their journey into client-side application logic.

\vspace{5mm}
\noindent
\section{WEEK 4 – FINAL PROJECT DEVELOPMENT: SCIENTIFIC WEB CALCULATOR}

The final week was devoted to consolidating all acquired skills through the development of a comprehensive web-based project. The assigned task involved creating a \textbf{Scientific Web Calculator} that integrated structural design, responsive styling, and functional JavaScript logic into a cohesive product.

\textbf{Project Highlights:}
\begin{itemize}
    \item Interface design using HTML and CSS for readability, accessibility, and mobile responsiveness.
    \item Implementation of arithmetic and scientific functions using JavaScript event handling.
    \item Creation of modular and reusable code using classes and methods to handle button clicks and display updates.
    \item Integration of DOM manipulation and data validation to ensure smooth user interaction.
    \item Development of additional interactive components such as counters and color selectors using event listeners and encapsulated class structures.
    \item Exploration of Bootstrap for improved UI consistency and rapid component styling.
    \item Construction of a mood tracker interface using a \texttt{MoodHandler} class to dynamically update emoji states, background color, and descriptive text.
    \item Application of switch statements and data attributes for cleaner, scalable event-driven logic.
    \item Incorporation of Font Awesome icons and Bootstrap styling to achieve a polished, professional appearance.
\end{itemize}

This project synthesized all major aspects of front-end development — structure, design, and interactivity. Learners demonstrated a complete development cycle from conceptualization to deployment, gaining hands-on experience in real-world application design and reinforcing the integrated relationship between HTML, CSS, and JavaScript.

\vspace{5mm}
\noindent
\section{TOOLS AND TECHNOLOGIES USED}

\begin{table}[H]
\centering
\caption{Tools and Technologies Used}
\begin{tabular}{|p{5cm}|p{8cm}|}
\hline
\textbf{Technology / Tool} & \textbf{Purpose / Usage} \\ \hline
HTML5 & Structure and layout of web content \\ \hline
CSS3 & Styling and presentation of web pages \\ \hline
JavaScript & Adding logic and interactivity \\ \hline
Visual Studio Code & Code editing and project organization \\ \hline
Google Chrome Developer Tools & Debugging and layout inspection \\ \hline
Git and GitHub & Version control and repository management \\ \hline
\end{tabular}
\end{table}

\vspace{3mm}
The training concluded with a comprehensive understanding of web development fundamentals. Although introductory in nature, it provided a solid technical foundation for further exploration into advanced concepts such as responsive design frameworks, client-server communication, and backend development.


\newpage
\chapter{RESULTS AND DISCUSSIONS}

\noindent
\section{Overview of the Project Output}

The final project developed during the industrial training was a \textbf{Scientific Web Calculator} built using \textbf{HTML}, \textbf{CSS}, and \textbf{JavaScript}. The objective of the project was to design and implement a responsive and user-friendly calculator capable of performing both basic arithmetic and fundamental scientific operations such as trigonometric, logarithmic, exponential, and square root calculations.

The project serves as a practical implementation of front-end web development concepts learned during the training period. It focuses on dynamic user interaction, functional computation through JavaScript logic, and a visually appealing, responsive interface designed using modern CSS styling principles.

\bigskip
\noindent
\section{Implementation Results}

The Scientific Calculator was implemented as a modular, browser-based application. The development process was divided into three main stages—structural design, styling, and scripting—each focusing on a specific technological component.

\bigskip
\noindent
\subsection{Interface Layout and Design}

The calculator interface was created using \textbf{HTML5} to define the logical structure of the application. The layout consists of two primary sections:
\begin{itemize}
    \item \textbf{Main Calculator Grid:} Includes numeric buttons (0–9), arithmetic operations (+, –, ×, ÷), and special keys such as \textit{AC}, \textit{DEL}, and \textit{=}.
    \item \textbf{Scientific Function Panel:} Contains buttons for trigonometric functions (\textit{sin}, \textit{cos}, \textit{tan}), logarithmic and exponential functions (\textit{log}, \textit{exp}), square root, and mathematical constants such as \textit{$\pi$}.
\end{itemize}

This structure ensures that both standard and scientific computations can be performed from a single, organized interface.

\bigskip
\noindent
\subsection{Visual Styling and Responsiveness}

The visual presentation of the calculator was developed using \textbf{CSS3}. The objective was to achieve a modern and intuitive appearance suitable for both desktop and mobile screens. 

\textbf{Key Features:}
\begin{itemize}
    \item \textbf{Dark Metallic Theme:} The background and button elements utilize dark gray and gradient tones to give a professional, high-contrast appearance.
    \item \textbf{Neon Green Display:} The calculator output screen is styled with neon green text on a dark background to simulate an LCD effect.
    \item \textbf{Hover and Click Effects:} Smooth transitions and shadow effects are used to indicate interactivity.
    \item \textbf{Responsive Layout:} The grid layout adapts to different screen sizes, ensuring usability across devices.
\end{itemize}

These design elements contribute to a polished user experience, maintaining both functionality and aesthetic appeal.

\bigskip
\noindent
\subsection{Functional Logic and Interactivity}

The computational logic was implemented using \textbf{JavaScript (ES6)}. The calculator operates through an object-oriented approach, encapsulated in a \texttt{Calculator} class responsible for handling user input, updating the display, and evaluating expressions.

\textbf{Core Features Implemented:}
\begin{itemize}
    \item \textbf{Input Handling:} Captures numeric, operational, and functional button presses using event listeners.
    \item \textbf{Expression Evaluation:} Mathematical expressions are preprocessed and evaluated using JavaScript’s \texttt{eval()} function with custom preprocessing for trigonometric and logarithmic operations.
    \item \textbf{Trigonometric Functions in Degrees:} Custom functions such as \texttt{sinDeg()}, \texttt{cosDeg()}, and \texttt{tanDeg()} ensure angle inputs are interpreted in degrees rather than radians.
    \item \textbf{Error Handling:} Displays “Error” for invalid expressions or undefined results to prevent unexpected behavior.
    \item \textbf{Dynamic Font Scaling:} The display font size adjusts automatically based on the length of the input or result.
\end{itemize}

This logic ensures smooth interaction, accurate computation, and a responsive user interface suitable for educational and demonstrative purposes.

\bigskip
\noindent
\subsection{Output Screens}

The following figures illustrate the functional output of the developed Scientific Calculator web application:

\begin{figure}[H]
    \centering
    \includegraphics[width=0.45\textwidth, keepaspectratio]{home_calc.png}
    \caption{Figure 3.1 – Home screen layout of the Scientific Calculator interface.}
    \label{fig:Calculator Home Screen}
\end{figure}

\begin{figure}[H]
    \centering
    \includegraphics[width=0.45\textwidth, keepaspectratio]{log_calc.png}
    \caption{Figure 3.2 – Execution of trigonometric and logarithmic operations.}
    \label{fig:Calculator Logarithmic Operations}
\end{figure}

\begin{figure}[H]
    \centering
    \includegraphics[width=0.45\textwidth, keepaspectratio]{error_calc.png}
    \caption{Figure 3.3 – Error handling and result display for invalid inputs.}
    \label{fig:Calculator Error Display}
\end{figure}


Each interface element was designed to contribute to user accessibility, operational clarity, and accurate computation.

\bigskip
\noindent
\section{Discussions and Observations}

Throughout the development process, the following key observations were made:
\begin{itemize}
    \item The separation of structure (HTML), style (CSS), and logic (JavaScript) proved essential for organized and maintainable code.
    \item Implementing trigonometric functions in degrees required custom conversions from degrees to radians, reinforcing understanding of mathematical transformations in JavaScript.
    \item Event-driven programming effectively handled user interaction and real-time display updates.
    \item While the calculator operates entirely on the client side, future enhancements could include persistent history logging or advanced symbolic computation.
\end{itemize}

Overall, the project demonstrated how foundational web technologies can be combined to create a fully functional and aesthetically refined application.

\bigskip
\noindent
\section{Summary}

This chapter presented the implementation results and observations for the Scientific Calculator web application. The calculator successfully performs both basic and scientific computations using a clean, interactive interface. The discussion highlighted the key aspects of design, functionality, and usability achieved during the development phase. The results affirm the effectiveness of using HTML, CSS, and JavaScript as introductory tools for practical web application development.


\newpage
\chapter{CONCLUSION}

The four-week training program in Web Development provided a structured and foundational introduction to modern web technologies. The sessions offered a systematic progression from fundamental concepts of HTML and CSS to the basic understanding of JavaScript, equipping trainees with the essential knowledge required to design and implement static and interactive web pages. 

Throughout the training, emphasis was placed on practical implementation and conceptual clarity. The exposure to real development environments, coding standards, and responsive design principles contributed to developing a strong groundwork for future learning. While the scope of the program was introductory, the clarity achieved in basic front-end development concepts created a robust base for deeper exploration into dynamic and full-stack web application development.

The final project—a scientific web calculator—served as a synthesis of the acquired concepts. It demonstrated the integration of HTML for structure, CSS for layout and visual presentation, and JavaScript for interactivity. Although modest in complexity, the project represented an important step toward understanding client-side logic and the functional potential of web applications.

In conclusion, the training successfully fulfilled its objective of introducing the core principles of web development to beginners. It provided not only theoretical comprehension but also practical competence, instilling confidence to independently pursue more advanced technologies and frameworks. The experience has laid a durable foundation for continued learning and professional growth in the evolving domain of web technologies.


% ------------------------------------------------
% REFERENCES
% ------------------------------------------------
\newpage
\begin{center}
    \Large\textbf{REFERENCES}
\end{center}

\begin{enumerate}
    \item W3C (World Wide Web Consortium). \textit{HTML Living Standard}. Available at: \url{https://html.spec.whatwg.org/} [Accessed July 2025].

    \item Mozilla Developer Network (MDN). \textit{HTML Documentation}. Available at: \url{https://developer.mozilla.org/en-US/docs/Web/HTML} [Accessed July 2025].

    \item Mozilla Developer Network (MDN). \textit{CSS: Cascading Style Sheets Documentation}. Available at: \url{https://developer.mozilla.org/en-US/docs/Web/CSS} [Accessed July 2025].

    \item Mozilla Developer Network (MDN). \textit{JavaScript Reference and Guide}. Available at: \url{https://developer.mozilla.org/en-US/docs/Web/JavaScript} [Accessed July 2025].

    \item Duckett, Jon. \textit{HTML and CSS: Design and Build Websites}. John Wiley and Sons, 2011.

    \item Duckett, Jon. \textit{JavaScript and JQuery: Interactive Front-End Web Development}. John Wiley and Sons, 2014.

    \item Robbins, Jennifer Niederst. \textit{Learning Web Design: A Beginner’s Guide to HTML, CSS, JavaScript, and Web Graphics}. 5th Edition, O’Reilly Media, 2018.

    \item Keith, Jeremy. \textit{HTML5 for Web Designers}. A Book Apart, 2010.

    \item Meyer, Eric A. \textit{CSS: The Definitive Guide}. 4th Edition, O’Reilly Media, 2017.

    \item Crockford, Douglas. \textit{JavaScript: The Good Parts}. O’Reilly Media, 2008.

    \item Flanagan, David. \textit{JavaScript: The Definitive Guide}. 7th Edition, O’Reilly Media, 2020.

    \item Bootstrap Team. \textit{Bootstrap Documentation}. Available at: \url{https://getbootstrap.com/docs/5.3/getting-started/introduction/} [Accessed July 2025].

    \item Visual Studio Code Documentation. \textit{User Guide and Extensions}. Microsoft Corporation. Available at: \url{https://code.visualstudio.com/docs} [Accessed July 2025].

    \item GitHub Documentation. \textit{Getting Started with Git and GitHub}. Available at: \url{https://docs.github.com/en/get-started} [Accessed July 2025].

    \item Khan Academy. \textit{Intro to HTML/CSS: Making Webpages}. Available at: \url{https://www.khanacademy.org/computing/computer-programming/html-css} [Accessed July 2025].

    \item freeCodeCamp. \textit{Responsive Web Design Certification}. Available at: \url{https://www.freecodecamp.org/learn/} [Accessed July 2025].

    \item W3Schools. \textit{HTML, CSS, and JavaScript Tutorials}. Available at: \url{https://www.w3schools.com/} [Accessed July 2025].

    \item Academic Advancement of Information Technology (A2IT), Mohali. \textit{Training Modules and Lecture Material on Web Development}, 2025.

    \item Guru Nanak Dev Engineering College, Ludhiana. \textit{Industrial Training Guidelines and Evaluation Criteria}, Department of Computer Science and Engineering, 2025.
\end{enumerate}

% ------------------------------------------------
% APPENDIX
% ------------------------------------------------
\appendix
\chapter*{APPENDIX}

\section*{A. Source Code}

The following C++ program represents the core functionality and logic design of the calculator project developed as part of this semester’s coursework. The focus is on computational efficiency and clean modular structure.

\begin{lstlisting}[language=C++, caption={Main Calculator Code}]
#include <bits/stdc++.h>
#include <chrono>
using namespace std;
using namespace std::chrono;

int main() {
    cout << "Scientific Calculator\n";
    double a, b;
    char op;
    cout << "Enter expression (e.g., 3 + 4): ";
    cin >> a >> op >> b;

    switch(op) {
        case '+': cout << "Result: " << a + b; break;
        case '-': cout << "Result: " << a - b; break;
        case '*': cout << "Result: " << a * b; break;
        case '/':
            if (b != 0)
                cout << "Result: " << a / b;
            else
                cout << "Error: Division by zero";
            break;
        default:
            cout << "Invalid Operator";
    }

    return 0;
}
\end{lstlisting}

The calculator integrates both basic and advanced functionalities, providing:

\begin{itemize}
    \item \textbf{Basic Function Panel:} Digits (0--9), arithmetic operations, clear, and backspace.
    \item \textbf{Scientific Function Panel:} Trigonometric (\textit{sin}, \textit{cos}, \textit{tan}), logarithmic, and exponential functions.
    \item \textbf{Result Display:} Displays evaluated results dynamically.
\end{itemize}

\section*{B. Experimental Results}

\begin{table}[H]
\centering
\begin{tabular}{|c|c|c|c|c|}
\hline
\textbf{Test Case} & \textbf{Input Expression} & \textbf{Expected Output} & \textbf{Observed Output} & \textbf{Remarks} \\ \hline
1 & 3 + 5 & 8 & 8 & Correct \\ \hline
2 & 7 / 0 & Error & Error & Correct error handling \\ \hline
3 & sin(90) & 1 & 1 & Verified \\ \hline
4 & log(1) & 0 & 0 & Correct \\ \hline
5 & exp(1) & 2.718 & 2.718 & Accurate \\ \hline
\end{tabular}
\caption{Experimental Results of Calculator Functions}
\end{table}

\section*{C. Execution Time Analysis}

\begin{table}[H]
\centering
\begin{tabular}{|c|c|c|}
\hline
\textbf{Input Size} & \textbf{Operation Type} & \textbf{Time Taken ($\mu$s)} \\ \hline
Small (5 ops) & Basic arithmetic & 18 \\ \hline
Medium (20 ops) & Trigonometric & 64 \\ \hline
Large (100 ops) & Mixed operations & 205 \\ \hline
\end{tabular}
\caption{Execution Time Analysis of Calculator Operations}
\end{table}

The observed time complexity shows near-linear growth with increased input size, confirming the efficiency of the program design.

\section*{D. Design Flow Diagram}

\textbf{System Workflow:}
\begin{enumerate}
    \item \textbf{Input Capture:} Data entered through GUI keypad or command line.
    \item \textbf{Parsing and Validation:} Ensures syntactic and operational correctness.
    \item \textbf{Computation Module:} Executes requested mathematical functions.
    \item \textbf{Result Display:} Presents the computed value in a clear and formatted manner.
\end{enumerate}

\section*{E. Tools and Technologies Used}
\begin{itemize}
    \item \textbf{Programming Language:} C++
    \item \textbf{Documentation Tool:} LaTeX
    \item \textbf{IDE:} Visual Studio Code
    \item \textbf{Compiler:} GNU GCC
    \item \textbf{Version Control:} Git and GitHub
\end{itemize}

\section*{F. Challenges Faced and Solutions}

\begin{table}[H]
\centering
\begin{tabular}{|p{3cm}|p{5cm}|p{5cm}|}
\hline
\textbf{Challenge} & \textbf{Description} & \textbf{Resolution} \\ \hline
GUI alignment & Difficulty maintaining uniform button spacing & Used \LaTeX{} tabular layouts for consistent grid design \\ \hline
Expression parsing & Handling operator precedence & Implemented modular switch-case structure for evaluation \\ \hline
Floating point errors & Inaccuracy in trigonometric outputs & Adopted \texttt{<cmath>} library for precision computation \\ \hline
\end{tabular}
\caption{Challenges and Resolutions During Development}
\end{table}

\section*{G. References}
\begin{enumerate}
    \item Cormen, T. H., et al. \textit{Introduction to Algorithms}, MIT Press.
    \item Stroustrup, B. \textit{The C++ Programming Language}, Addison-Wesley.
    \item Official GNU GCC Documentation.
    \item Online \LaTeX{} Project Public License (LPPL).
    \item W3Schools and GeeksforGeeks (syntax references and implementation ideas).
\end{enumerate}

\end{document}